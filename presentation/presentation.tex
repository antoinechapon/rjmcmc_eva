\documentclass[aspectratio=169]{beamer}


\mode<presentation>
{
  \usetheme{Antibes}

  \setbeamercovered{transparent}
}



\usepackage[french]{babel}
\usepackage[utf8]{inputenc}
\usepackage{times}
\usepackage[T1]{fontenc}
\usepackage{xcolor}
\usepackage{booktabs}
\usepackage{threeparttable}
\usepackage{tabularx}
\usepackage{caption}
\usepackage{numprint}
\usepackage{amsmath,amsfonts,amsthm,bm}
\usepackage[export]{adjustbox}
\usepackage{tikz}
\usepackage{float}
\usepackage{listings}
\usepackage{textcomp}
\usepackage[backend=biber, style=authoryear, defernumbers=true]{biblatex}

\addbibresource{SIG_M2.bib}

\renewcommand*{\bibfont}{\scriptsize}

\captionsetup{font=scriptsize, labelfont=scriptsize}

\date{janvier 2021}

%\usefonttheme[onlymath]{serif}

\definecolor{my_blue}{rgb}{0.3, 0.4, 0.5}
\definecolor{my_grey}{rgb}{0.6, 0.6, 0.6}

\usecolortheme[named = my_blue]{structure}

\setbeamertemplate{footline}{}

\setbeamerfont{author in head/foot}{size=\fontsize{4}{4.8}\selectfont}

\setbeamertemplate{bibliography item}{}
\setbeamertemplate{bibliography entry article}{}
\setbeamertemplate{bibliography entry title}{}
\setbeamertemplate{bibliography entry location}{}
\setbeamertemplate{bibliography entry note}{}


\setbeamerfont{bibliography entry author}{size=\footnotesize}
\setbeamerfont{bibliography entry title}{size=\footnotesize}
\setbeamerfont{bibliography entry location}{size=\footnotesize}
\setbeamerfont{bibliography entry note}{size=\footnotesize}

\defbeamertemplate{footline}{myframe number}
{
  \hfill%
  \usebeamercolor[fg]{page number in head/foot}%
  \usebeamerfont{page number in head/foot}%
  \raisebox{0.3cm}[0pt][0pt]{% <--- change here
    \insertframenumber\,/\,\inserttotalframenumber\kern1em}%
}

\setbeamertemplate{footline}[myframe number]
\beamertemplatenavigationsymbolsempty

\newcommand\Wider[2][3em]{%
\makebox[\linewidth][c]{%
  \begin{minipage}{\dimexpr\textwidth+#1\relax}
  \raggedright#2
  \end{minipage}%
  }%
}


\title{Évolution des classes d\rq{}occupation des sols par bassin versant dans la région d\rq{}Austin (Texas)}

\author{Antoine~Chapon~et Alexandre~Ratel}

\institute[Universities of Somewhere and Elsewhere] % (optional, but mostly needed)
{
  Master 2 Gestion Durable des Hydrogéosystèmes\\
  Université de Rouen
 }
% - Use the \inst command only if there are several affiliations.
% - Keep it simple, no one is interested in your street address.

%\date[CFP 2003] % (optional, should be abbreviation of conference name)
%{Conference on Fabulous Presentations, 2003}
% - Either use conference name or its abbreviation.
% - Not really informative to the audience, more for people (including
%   yourself) who are reading the slides online

%\subject{Theoretical Computer Science}
% This is only inserted into the PDF information catalog. Can be left
% out. 



% If you have a file called "university-logo-filename.xxx", where xxx
% is a graphic format that can be processed by latex or pdflatex,
% resp., then you can add a logo as follows:

% \pgfdeclareimage[height=0.5cm]{university-logo}{university-logo-filename}
% \logo{\pgfuseimage{university-logo}}



% Delete this, if you do not want the table of contents to pop up at
% the beginning of each subsection:
\AtBeginSubsection[]
{
\begin{frame}<beamer>{Plan}
\frametitle{Plan}
\tableofcontents[currentsection,currentsubsection]
\end{frame}
}


% If you wish to uncover everything in a step-wise fashion, uncomment
% the following command: 

%\beamerdefaultoverlayspecification{<+->}


\begin{document}

\begin{frame}
  \titlepage
\end{frame}



\section*{}

\begin{frame}{Introduction}
	\begin{itemize}
	\setlength{\itemsep}{10pt}
	\item Occupation du sol (\emph{land use} -- LU).
	\item Changement de LU $\rightarrow$ potentielle imperméabilisation.
	\item Processus anthropique mais conditionné par des paramètres physiques.
	\item La pente est un paramètre important.
	\end{itemize}
\vspace{0.5cm}
Changement de LU dans 2 bassins versants (BV) d\rq{}Austin, en lien avec le relief.\\
\vspace{0.5cm}
		{\scriptsize
		\cite{united_states_environmental_protection_agency_land_2017, exum_estimating_2005, fu_temporal_2006}}
\end{frame}

\begin{frame}{Plan}
\tableofcontents
\end{frame}


\section{Site d\rq{}étude et données}

\subsection{Localisation des bassins versants dans l\rq{}aire urbaine d\rq{}Austin}

\begin{frame}
\begin{columns}
	\begin{column}{0.75\textwidth}
		\begin{figure}
	 		\includegraphics[height=0.95\textheight, center]{intro_map.png}
		\end{figure}
	\end{column}
	\begin{column}{0.25\textwidth}
             	Zoom avec \emph{overview}.\\
		\vspace{1cm}
		{\scriptsize
		\cite{united_states_census_bureau_census_2020, united_states_census_bureau_us_2021}}
	\end{column}
\end{columns}
\end{frame}

\subsection{Données d\rq{}occupation des sols}

\begin{frame}
\begin{columns}
	\begin{column}{0.5\textwidth}
		\begin{itemize}
		\setlength{\itemsep}{10pt}
		\item LU fournies pour 1990, 1995, 2000 et 2003.
		\item Autres données disponibles pour 2010 et une date non-spécifiée.
		\item Polygones avec classe de LU.
		\item Données partiellement ou totalement inutilisables.
		\end{itemize}
	\end{column}
	\begin{column}{0.5\textwidth}
		\begin{itemize}
		\setlength{\itemsep}{10pt}
		\item Extraction des parcelles dans les BVs.
		\item Recalcul de l\rq{}aire des parcelles.
		\item Jointure de toutes les données de LU aux parcelles de 2010.
		\item Polygones des routes absents en 2010 : calcul de la différence avec le  BV et fusion avec les autres parcelles.
		\end{itemize}
	\end{column}
\end{columns}
	\vspace{1cm}
	{\scriptsize
	\cite{city_of_austin_planning_and_development_review_2010_2021, city_of_austin_planning_and_development_review_land_2021}}
\end{frame}

\subsection{Données de topographie}

\begin{frame}
	\begin{figure}
	 	\includegraphics[width=1.1\textwidth, center]{MNT_correction.png}
	\end{figure}
Décalage estimé par outils de mesure, puis correction des coordonnées ($‐50$ m en X et $+200$ m en Y).
\end{frame}


\section{Occupation des sols et urbanisation}

\subsection{Méthode d\rq{}analyse de l\rq{}évolution de l\rq{}urbanisation}

\begin{frame}{Classes d\rq{}occupation des sols}
\begin{table}[ht]
\centering
\scriptsize
\begin{tabularx}{\textwidth}{lllll}
  \toprule
indice urbanisation & classe simple & nom classe & classe complète & définition \\ 
  \midrule
  0 &   7 & espace verts & 700 & ? \\ 
     &    &  & 710 & Parks/Greenbelts  \\ 
     &    &  & 720 & Golf Courses  \\ 
\vdots &  \vdots & \vdots & \vdots & \vdots \\
     &   9 & agricole ou naturel & 910 & Agricultural \\ 
     &    &  & 940 & Water  \\ 
     &    &  & 999 & Unknown \\ 
     \midrule
\vdots &  \vdots & \vdots & \vdots & \vdots \\
     \midrule
    3 &   5 & industrie & 500 & ? \\ 
\vdots &  \vdots & \vdots & \vdots & \vdots \\
     &   8 & route ou transport & 800 & ? \\ 
\vdots &  \vdots & \vdots & \vdots & \vdots \\
   \bottomrule
\end{tabularx}
\end{table}
\end{frame}

\begin{frame}{Indice d\rq{}urbanisation}
\begin{columns}
	\begin{column}{0.5\textwidth}
	\begin{block}{Calcul de champ}
	CASE\\
	WHEN LU1990 in (7, 9) THEN 0\\
	WHEN LU1990 in (0, 1, 3) THEN 1\\
	WHEN LU1990 in (2, 4, 6) THEN 2\\
	WHEN LU1990 in (5, 8) THEN 3\\
	END
	\end{block}
	\end{column}
	\begin{column}{0.5\textwidth}
		\begin{itemize}
		\setlength{\itemsep}{10pt}
		\item \lq\lq{}potentiel de ruissellement\rq\rq{} de la classe de LU.
		\item Indice ordonné.
		\item Permet de calculer une évolution.
		\end{itemize}
	\end{column}
\end{columns}
\end{frame}

\subsection{Évolution de l\rq{}occupation des sols de 1990 à 2010}

\begin{frame}
\begin{columns}
	\begin{column}{0.75\textwidth}
	\begin{figure}
	 	\includegraphics[height=0.97\textheight, center]{LandUse_Bull.png}
	\end{figure}
	\end{column}
	\begin{column}{0.25\textwidth}
	\begin{figure}
	 	\begin{itemize}
		\item Résultats présentés uniquement pour le BV de Bull.
		\item Pas de routes en 1990 ?
		\item Encore plus de problèmes pour le BV de Williamson.
		\end{itemize}
	\end{figure}
	\end{column}
\end{columns}
\end{frame}

\begin{frame}{Évolution de l\rq{}occupation pour le BV de Bull}
\begin{columns}
	\begin{column}{0.77\textwidth}
		 \includegraphics[height=0.8\textheight, center]{LUbull_area.pdf}
	\end{column}
	\begin{column}{0.23\textwidth}
		 \begin{itemize}
		\item Diminution classe 9.
		\item Augmentation classes 7 et 1.
		\item Stabilité classe 8.
		\item Disparition de la classe 0 en 2003, impact ?
		\end{itemize}
	\end{column}
\end{columns}
\end{frame}

\begin{frame}{Évolution de l\rq{}indice d\rq{}urbanisation pour le BV de Bull}
\includegraphics[width=1.13\textwidth ,center]{DeltaUrb.png}
\end{frame}


\section{Relation entre topographie, pente et occupation des sols}

\subsection{Méthodes d\rq{}analyse des données topographiques}

\begin{frame}{Création d\rq{}une couche avec LU, élévation et pente}
	\begin{enumerate}
	\setlength{\itemsep}{10pt}
	\item Raster de pente créé à partir du MNT avec la fonction \emph{slope}.
	\item Rasters de MNT et pente convertis en points.
	\item Extraction des données de LU avec les points de MNT.
	\item Jointure des points de LU, MNT et pente.
	\item Couche exportable pour analyse dans R.
	\end{enumerate}
\end{frame}

\subsection{Impact de la topographie sur l\rq{}occupation des sols et l\rq{}urbanisation}

\begin{frame}{Topographie et pente du BV de Bull}
\includegraphics[width=\textwidth, center]{topo_Bull.png}
\end{frame}

\begin{frame}{Lien \emph{local} entre relief et LU}
\includegraphics[width=0.93\textwidth, center]{habitat_MNT.png}
\end{frame}

\begin{frame}{Topographie et pente du BV de Williamson et courbe hypsométrique}
\begin{columns}
	\begin{column}{0.65\textwidth}
	\includegraphics[height=0.87\textheight, center]{topo_Williamson.png}
	\end{column}
	\begin{column}{0.35\textwidth}
	\includegraphics[height=0.5\textheight, center]{courbe_hypso.pdf}
	\end{column}
\end{columns}
\end{frame}

\begin{frame}{Lien entre élévation et LU}
\includegraphics[width=0.95\textwidth, center]{boxplot_MNT.pdf}
\end{frame}

\begin{frame}{Lien entre pente et LU}
\includegraphics[width=0.95\textwidth, center]{boxplot_slope.pdf}
\end{frame}

\begin{frame}{Lien entre relief et la classe 5 \lq\lq{}industrie\rq\rq{}}
\includegraphics[width=1.13\textwidth, center]{indus_slope.png}
\end{frame}


\section{Conclusion}

\begin{frame}<beamer>{Plan}
\frametitle{Plan}
\tableofcontents[currentsection]
\end{frame}

\begin{frame}
 	\begin{itemize}
 	\setlength{\itemsep}{10pt}
	\item Augmentation du degré d\rq{}urbanisation des BVs ...
	\item ... principalement le long des routes majeures.
	\item Pas de lien entre LU et altitude.
	\item Quelques relations entre LU et pente.
	\item Le réseau de transport semble plus déterminant que le relief ...
	\item ... mais ce réseau pourrait aussi être dépendant du relief.
	\item Plus de données devraient être analysées pour réellement conclure.
	\end{itemize}
\end{frame}

\section*{}

\begin{frame}[plain]{Références}
\printbibliography
\end{frame}

{
\setbeamercolor{background canvas}{bg=my_blue}
\begin{frame}[plain]
\begin{tikzpicture}[overlay, remember picture]
\node[anchor=center] at (current page.center) {
\begin{beamercolorbox}[center]{title}
     \centerline{\huge{\textcolor{white}{merci pour votre attention}}}
  \end{beamercolorbox}};
\end{tikzpicture}
\end{frame}
}


\end{document}


